\documentclass[a4paper,11pt]{article}

\usepackage[utf8]{inputenc}
\usepackage[T1]{fontenc}
\usepackage[english]{babel}
\usepackage{amsmath,amssymb}
\usepackage{graphicx}
\usepackage{hyperref}

\title{Radio Signal Classification}
\author{Your Name}
\date{\today}

\begin{document}
\maketitle

\section{Introduction}

We consider a signal $x(t)$ defined as
\[
x(t) = A \cos(2 \pi f_0 t + \phi),
\]
where $A$ denotes the amplitude, $f_0$ the carrier frequency, and $\phi$ the phase.

\section{Illustrative Figure}

\begin{figure}[h]
  \centering
  \includegraphics[width=.7\textwidth]{figures/fig1.pdf}
  \caption{Amplitude spectrum of a radio-frequency signal.}
  \label{fig:spectrum}
\end{figure}

As shown in Figure~\ref{fig:spectrum}, the spectral content is concentrated around the carrier frequency.

\section{Conclusion}

Additional sections may introduce modulation schemes, bandwidth considerations, or classification architectures.

\end{document}
